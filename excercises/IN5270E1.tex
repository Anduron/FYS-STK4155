\documentclass{article}


\usepackage[utf8]{inputenc}
\usepackage{graphicx}
\usepackage{float}
\usepackage{amsmath}
\usepackage{epsfig,floatflt}
\usepackage{url}
\usepackage{hyperref}

\title{IN5270 Exercise 1}

\author{Stig-Nicolai Foyn \\ email \href{mailto:stignicf@student.matnat.uio.no}{stignicf@student.matnat.uio.no}}

\maketitle

\begin{document}

%\date{Received - / Accepted -}

\section{Exercise (1a):}
\label{sec:1a}
We want to solve the ODE below numerically and analytically:
%
\begin{equation*}
    u'' + \omega^2 u = f(t), \quad u(0) = I, \quad u'(o) = V, \quad t\in (0,T]
\end{equation*}
%
We start by discretizing the equation, with n as points in time:

\begin{equation}
    u''(t) = \frac{u_{n+1} -2u_{n} + u_{n-1}}{\Delta t^2}
\end{equation}

which leads to:

\begin{equation*}
    \frac{u_{n+1} -2u_{n} + u_{n-1}}{\Delta t^2} + \omega^2 u_{n} = f(t_n)
\end{equation*}

And lastly we solve for $u_{n+1}$:

\begin{equation*}
    u_{n+1} = 2u_{n} - u_{n-1} \Delta t^2(f(t_n) - \omega^2 u_{n})
\end{equation*}

We need to satisfy the initial conditions and find the equation at $n=0$, which is the first timestep:

\begin{equation*}
    u_{1} = 2u_{0} - u_{-1} \Delta t^2(f(t_0) - \omega^2 u_{0})
\end{equation*}

Now it is necessary to find $u_{-1}$, which can be done by ($[D_{2t}u]_n$ for $n=0$):

\begin{equation}
    u'(n) \approx \frac{u_{n+1} - u_{n-1}}{2\Delta t}
\end{equation}

Which we know at $n=0$ gives:

\begin{equation*}
    u'(0) \approx \frac{u_{1} - u_{-1}}{2\Delta t} = V
\end{equation*}

Thus:

\begin{equation*}
    u_{-1} = u_{1} - 2V\Delta t
\end{equation*}

Now the first timestep becomes:
%
\begin{align}
    u_{1} &= 2u_{0} - u_{1} + 2V\Delta t + \Delta t^2(f(t_0) - \omega^2 u_{0})\nonumber \\
    &= u_{0} + V\Delta t + \frac{1}{2}\Delta t^2(f(t_0) - \omega^2 u_{0})\nonumber\\
    &= I + V\Delta t + \frac{1}{2}\Delta t^2(f(t_0) - \omega^2 I)
\end{align}
%
We now have the first timestep expressed as an equation containing constants we "know" and the function $f(t)$ which we will soon find.

\newpage

\section{Exercise (1b):}
\label{sec:1b}
We now "guess" a solution $u_e(x,t) = ct + d$, and use the intial condtions:

\begin{equation*}
    u_e'(0) = c = u'(0) \quad \implies \quad c=V
\end{equation*}
\begin{equation*}
    u_e(0) = d = u(0) \quad \implies \quad d=I
\end{equation*}

\begin{equation}
    u_e = Vt + I
\end{equation}

Now we show that, by just inserting t into the equation, we get 0:

\begin{equation*}
    [D_tD_t t]_n = \frac{t_{n+1} - 2t_{n} + t_{n-1}}{\Delta t^2} = \frac{\Delta t - \Delta t}{\Delta t^2} = 0
\end{equation*}
Now we use that the operator $D_tD_t$ is linear and with $f(t) = \omega^2(Vt + I)$ get:
\begin{equation*}
    [D_tD_t u + \omega^2 u - f]_n = 0
\end{equation*}
\begin{align*}
    [D_tD_t (Vt +I) + \omega^2 (Vt + I) - f]_n
    &= [D_tD_t (Vt +I)]_n \\
    &= V[D_tD_t t]_n + [D_tD_t I]_n \\
    &= 0
\end{align*}
Which implies that $u_e$ is a perfect solution of the discrete equations

\section{Exercise (1c):}
\label{sec:1c}
%
The program returns the following results for the linear case:
\newline
\newline
Initial conditions u(0)=I, u'(0)=V:
\newline
residual step1: 0
\newline
residual: 0
\newline
\newline
This means that the solution is perfect (since the residual is zero).
%

\section{Exercise (1d):}
\label{sec:1d}
After extending the program we see that returns the following results for the quadratic $bt^2 + Vt + I$ case:
\newline
\newline
Initial conditions u(0)=I, u'(0)=V:
\newline
residual step1: 0
\newline
residual: 0
\newline
\newline
This means that the solution is perfect (since the residual is zero).

\section{Exercise (1e):}
\label{sec:1e}
After extending the program further we see that returns the following results for the qubic $at^3 + bt^2 + Vt + I$ case:
\newline
\newline
Initial conditions u(0)=I, u'(0)=V:
\newline
residual step1: a*dt**3
\newline
residual: 0
\newline
\newline
This means that we make an error $a\Delta t^3$ in the first timestep when we use a cubic function. Meaning the polynomial does not completely (the rest of the terms are 0) fulfill the discrete equations.


\section{Exercise (1f), (1g):}
\label{sec:1f}
Our program now comes with a simple numerical solver and will test the solver for the qubic case. This test will in turn return an "AssertionError" if the value of the error in our numerical solver is larger than our tolerance $\epsilon = 10^{-14}$. The numbers used in the test are made so that the test will not return "AssertionError", but still show the cubic trend in a plot of $u_e$ versus the numerical solution.

\begin{figure}[H]
\mbox{\epsfig{figure=nosetest.pdf,scale=0.65,clip=}}
\caption{Plot of the numerical and exact solution in the nosetest for a quadratic function.}
\label{fig:fig5}
\end{figure}

\end{document}
